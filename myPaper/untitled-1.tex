\documentclass[12pt]{article}

\begin{document}
\title{A Novel Application of Computer Vision and  Machine Learning for Three-Dimensional Reconstruction of Lung Anatomy and Pathology}

% If all authors have the same affiliation do not use the \inst
% command: 
\author{Done by: Bardh Rushiti, Mentored by: Dr. Thomas Kinsman}

% Short affiliations to be placed on the first page
% Please give the name of your institution e.g. a university or an
% institute (if not part of a university).
%\affiliation{ Rochester Institute of Technology}

%\keywords{pulmonary nodule, detection, segmentation, \\3d reconstruction}

\maketitle

\begin{abstract}
This project presents a pipeline for the three dimensional reconstruction of lung anatomy and pathology using advanced computer vision and machine learning techniques. The pipeline consists of several stages including image pre-processing, feature extraction, and model training and evaluation. The computer vision component detects and segments lung nodules within CT scans, while the machine learning component trains models for nodule detection. A 3D reconstruction module is also included, creating detailed 3D models of the lung, airways, and potential nodules, which can be analyzed for signs of malignancy based on nodule shape, location, and proximity to airways. The ultimate goal of the project is to improve the accuracy and efficiency of lung nodule detection and lung and airway reconstruction, with the potential to enhance the diagnostic capabilities of oncologists and improve patient outcomes.
\end{abstract}

\section{Introduction}
Lung cancer is the leading cause of cancer death worldwide, accounting for nearly a quarter of all cancer deaths [1]. Nonetheless, the detection and diagnosis of lung cancer remain a significant challenge in the medical field. Early detection is crucial for successful treatment, but the identification of lung nodules, small growths on the lung tissue, can be difficult due to their size and location.  \\
Computed Tomography (CT) scans are commonly used for detecting lung nodules, but manually identifying these nodules in CT scans can be a lengthy, laborious, and subjective process for radiologists. Accurate and precise segmentation and three dimensional (3D) reconstruction of the lung anatomy and pathology can provide more detailed information about their shape, size, and rate of change, which can be useful for diagnosis and treatment planning. \\
In recent years, computer vision [2, 3, 4] and machine learning techniques [5, 6, 7] have been applied in medical imaging to improve diagnostic accuracy and efficiency. This project focuses on utilizing these technologies to develop a pipeline for the 3D reconstruction of lung anatomy, lung airways, and potential malignant cancer. The pipeline consists of several stages including image pre-processing, feature extraction, model training, and evaluation, and 3D reconstruction.\\
The computer vision component is responsible for detecting and segmenting lung anatomy within CT scans. This stage involves image pre-processing techniques such as image normalization, filtering, and segmentation. The machine learning component uses 3D Deep Convolution Neural Networks (DCNN) trained in multi-task fashion models for nodule segmentation and false-positive reduction. \\
Finally, the 3D reconstruction module creates detailed 3D models of the lung, airways, and potential nodules, which can be analyzed for signs of malignancy based on nodule shape, size, location, and proximity to airways. This stage utilizes computer graphics and 3D reconstruction algorithms to create realistic models of the lung and airways. \\
The ultimate goal of this project is to improve the accuracy and efficiency of lung nodule detection and lung and airway reconstruction, with the potential to enhance the diagnostic capabilities of oncologists and improve patient outcomes.


\section{Motivation}
THis is a sample text.
This is the second lind.
This is the third line.

\end{document}